\section{Hauptfaser- und Rahmenbündel}
\label{sec:vektorbundel}

\thmbox{defn}{Hauptfaserbündel}{\label{def:hfb}Sei $(P, \pi, B)$ ein Faserbündel und $G$ eine Lie-Gruppe. $(P, \pi, B) = (P, \pi, B ; G)$ heißt \textbf{$G$-Hauptfaserbündel}, falls gilt:
\begin{enumerate}
 \item $G$ wirkt von rechts als Liesche Transformationsgruppe auf $P$. Die Wirkung ist frei, fasertreu und faserweise transitiv.
 \item \label{it:bundelatlas}Es gibt einen Bündelatlas $\{ \Phi_U \}$ aus G-äquivarianten (lokalen) Trivialisierungen. Das heißt:
      \begin{enumerate}
       \item $\Phi_U \colon \pi^{-1}(U) \to U \times G$ ist ein Diffeomorphismus.
       \item $\pi_U \after \Phi_U = \pi$ \footnotemark.
       \item $\Phi_U (p \cdot g) = \Phi_U(p) \cdot g$ für alle $p \in \pi^{-1}$, $g \in G$.
      \end{enumerate}\footnotetext{$\pi_U$ ist die Projektion $U \times G \to U$.}
\end{enumerate}
Es gilt also folgendes Schema:
\begin{center}
  \begin{tikzpicture}[->, >=stealth', shorten >=1pt, auto, node distance=2.8cm, semithick]
    \tikzstyle{every state}=[draw=none]

    \node[state] (A)			{$\pi^{-1}(U) \times G$};
    \node[state] (B) [below right of=A]	{$U \times G \times G$};
    \node[state] (C) [above right of=B] {$\pi^{-1}(U)$};
    \node[state] (D) [below right of=C] {$U \times G$};

    \path (A) edge      node {$\cdot_G$} 		(C)
	  (C) edge      node {$\pi_U$} 			(D)
	  (A) edge      node {$\Phi_U \times id_G$} 	(B)
	  (B) edge	node {$id_U \times \cdot_G$}	(D);
  \end{tikzpicture}
\end{center}%\\
Die Lie-Gruppe $G$ heißt auch \textbf{Strukturgruppe} des Hauptfaserbündels.
}

\thmbox{bem}{}{Für alle $p \in P$ ist $L_p \colon G \to P_{\pi(p)}, g \mapsto p \cdot g$ ein Diffeomorphismus. Daher ist die Lie-Gruppe $G$ die typische Faser von $(P, \pi, B)$.

Trotz dieser natürlichen Diffeomorphie existiert im Allgemeinen keine Gruppenstruktur auf den Fasern von $P$.}

\subsection{Schnitte}

Sei $(P, \pi, B)$ ein Hauptfaserbündel. Sei $U \subset B$ eine offene Umgebung und setze $P|_U := \pi^{-1}(U)$. Ferner sei eine lokale Trivialisierung \[\Phi_U \colon P|_U \to U \times G\] gegeben.
Dann kann wie folgt ein glatter Schnitt $s \colon U \to P|_U$ definiert werden: \[s(x) := \Phi_U^{-1}(x,1_G).\]
Für beliebiges $g \in G$ und $u \in U$ erhält man somit: \[\Phi_U^{-1}(u,g) = \Phi_U^{-1}(u,1_g \cdot g) = \Phi_U^{-1}(u,1_G) \cdot g = s(u) \cdot g\]

Sei umgekehrt ein glatter Schnitt $s \colon U \to P|_U$ gegeben. Dann existiert für jedes $p \in P_x, \ x \in B$ ein eindeutig bestimmtes Element $g_p \in G$, sodass \[p = s(\pi(p)) \cdot g_p\] gilt,
da die Gruppenwirkung frei und transitiv ist.
Durch die Zuweisung \[\Phi_U(p) := (\pi(p), g_p)\] erhält man so eine lokale Trivialisierung $\Phi_U \colon P|_U \to U \times G$.

\thmbox{flg}{}{Es existiert eine eins-zu-eins-Beziehung zwischen lokalen Trivialisierungen und lokalen Schnitten.}

Hauptfaserbündel können lokal also auch durch ihre Schnitte beschrieben werden. Dies soll im Folgenden konkret festgehalten werden:

\thmbox{lem}{}{\label{lem:altdef} Alternativ kann \autoref{it:bundelatlas} in \hyperref[def:hfb]{Definition }\ref{def:hfb} mit dem zuvor erläuterten auch folgendermaßen formuliert werden:
\begin{enumerate}
 \item[\ref*{it:bundelatlas}'.] Es existieren eine offene Überdeckung $\{ U_i \}_{i \in I}$ von $B$ und lokale Schnitte \linebreak $\{ s_i \colon U_i \to P \}_{i \in I}$, sodass
 \[\Psi_{s_i} \colon P_{U_i} \to G, p = s_i(\pi(p)) \cdot g_p \mapsto g_p\] glatt ist.
\end{enumerate}}

Nachdem wir nun eine alternative Charakterisierung für Hauptfaserbündel gefunden haben, muss noch definiert werden, wann Hauptfaserbündel isomorph sind. Anschließend wird erläutert, warum sich diese
Definition von der Isomorphie von Faserbündeln unterscheidet.

\thmbox{defn}{Isomorphie}{\label{def:iso}Zwei $G$-Hauptfaserbündel $(P, \pi, B)$ und $(\tilde P, \tilde \pi, B)$ über dem selben Basisraum $B$ heißen isomorph, falls es einen $G$-äquivarianten 
Diffeomorphismus 
$\Phi \colon P \to \tilde P$ gibt, für den $\tilde \pi \after \Phi = \pi$ gilt.}

\thmbox{bem}{}{\label{bem:equi}Die $G$-Äquivarianz ist wesentlich! Es gibt $G$-Hauptfaserbündel, die als Faserbündel betrachtet, nicht jedoch als Hauptfaserbündel isomorph sind.

Ein Gegenbeispiel ist leicht konstruiert, indem man bei identischen Faserbündeln die Gruppenwirkungderart definiert, dass $\Phi$ nicht G-äquivarianten ist. Die Faserbündel bleiben identisch, also 
insbesondere isomorph, als $G$-Hauptfaserbündel sind sie allerdings nicht isomorph.}

\hyperref[lem:altdef]{Lemma }\ref{lem:altdef} und \hyperref[def:iso]{Definition }\ref{def:iso}, insbesondere mit der vorigen \hyperref[bem:equi]{Bemerkung }\ref{bem:equi}, implizieren folgende 
Äquivalenz:

\thmbox{flg}{Trivialität}{Ein $G$-Hauptfaserbündel $(P, \pi, B)$ ist genau dann trivial, wenn es einen globalen Schnitt besitzt.}

\subsection{Konstruktionen}

\thmbox{lem}{Rückzug}{Sei $(P, \pi, B)$ ein $G$-Hauptfaserbündel und sei $F \colon B \to \tilde B$ eine glatte Abbildung. Das durch den Rückzug (vgl. \hyperref[def:ruck]{Definition }\ref{def:ruck})
konstruierte Faserbündel $(f^* P, \tilde \pi, \tilde B)$ ist bereits ein $G$-Hauptfaserbündel.}

\TODO{Beweis}

Im Folgenden soll eine Möglichkeit aufgezeigt werden, die typische Faser und damit die Lie-Gruppe eines Hauptfaserbündels zu ``ersetzen''. Sei dazu ein $G$-Hauptfaserbündel $(P, \pi, B)$ sowie
eine Mannigfaltigkeit $F$ und eine Transformationsgruppe $[F,G]$ gegeben.

Dann wirkt $G$ durch \[(p,f) \cdot g := (p \cdot g, g^{-1} \cdot f)\] von rechts auf $P \times F$. Im Folgenden bezeichnet $P \times_G F := \fak{P \times F}{G}$ den Faktorraum und
$[p,f]$ die Äquivalenzklasse von $(p,f)$. Damit wird folgende Abbildung definiert:\[\pi_{\times_G} \colon P \times_G F \to B, \ [p,f] \mapsto \pi(p)\]

\thmbox{lemdef}{Assoziiertes Faserbündel}{Das Tupel $(P \times_G F, \pi_{\times_G}, B)$ ist ein Faserbündel über $B$ mit typischer Faser $F$. Wir nennen es das zu $(P,\pi,B)$ und der 
Transformationsgruppe $[F,G]$ \textbf{assoziierte Faserbündel}.}

\TODO{Beweis}

Ist statt einer Transformationsgruppe $[F,G]$ ein Lie-Gruppen-Homomorphismus $\varphi \colon G \to F$ gegeben, so findet sich auch die Schreibweise $P \times_\varphi F$ als Alternative zu 
$P \times_G F$.

\subsubsection{Spezialfälle assoziierter Bündel}

\thmbox{bsp}{Assoziiertes Vektorbündel}{Seien $(P, \pi, B)$ ein $G$-Hauptfaserbündel, $V$ ein endlich-dimensionaler $\K$-Vektorraum und $\rho \colon G \to Aut(V)$ eine Darstellung von $G$. 
Dann ist $(P \times_\rho V, \pi_{\times_\rho}, B)$ ein $\K$-Vektorbündel mit typischer Faser $\K^n$ und wird das zu $(P, \pi, B)$ und $\rho$ \textbf{assoziierte Vektorbündel} genannt.}

\thmbox{bsp}{Erweiterung und Reduktion}{Ist $H$ eine Lie-Gruppe und $[H,G]$ eine Transformationsgruppe bzw. $\varphi \colon G \to H$ ein Lie-Gruppen-Homomorphismus, so ist 
$(P \times_G H, \pi_{\times_G}, B)$ bzw $(P \times_\varphi H, \pi_{\times_\varphi}, B)$ ein $H$-Hauptfaserbündel.
\begin{itemize}
 \item Gilt $H \unrhd G$ und $\varphi$ ist eine Einbettung, so wird auch von einer ($\varphi$-)\textbf{Erweiterung} der Strukturgruppe gesprochen.
 \item Gilt umgekehrt $H \unlhd G$, so wird auch von einer ($\varphi$-)\textbf{Reduktion} der Strukturgruppe gesprochen. Dies kann besonders hilfreich sein, wenn Eigenschaften der Strukturgruppe
 verschärft werden sollen. Beispielweise gibt es Aussagen, die eine Kompakte Strukturgruppe voraussetzen. Dies kann ggf. über eine solche Reduktion erreicht werden.
\end{itemize}}

\subsubsection{Rahmenbündel}

Sei $(V,\pi,B)$ ein $\K$-Vektorbündel von Rang $n$. Dann ist für alle $b \in B$ die Faser $V_b$ ein $n$-dimensionaler Vektorraum.

Setze nun $\mathfrak{B}_b := \{ (b_1, \dots , b_n) \in V_b \ | \ (b_1, \dots, b_n) \text{ ist eine (geordnete) Basis von } V_b \}$
Dann wirkt $GL(n,\K)$ durch \[ (b_1, \dots, b_n) \cdot A := \left( \sum_{i=1}^n A_{i1} b_i, \dots , \sum_{i=1}^n A_{in}b_n \right) \] frei und transitiv von rechts auf $\mathfrak{B}_b$.
Damit wird $(\mathfrak{B}, \tilde \pi, B)$ mit $\mathfrak{B} := \mathfrak{B}_V := \bigcup_{b \in B} \mathfrak{B}_b$ und $\tilde \pi \colon \mathfrak{B} \to B$, sodass $\tilde \pi|_{\mathfrak{B}_b} \equiv b$ gilt, zu
einem $GL(n,\K)$-Hauptfaserbündel.

\thmbox{defn}{Rahmenbündel}{Sei $B$ eine $n$-dimensionale glatte Mannigfaltigkeit und $V := TB \to B$ das Tangentialbündel. Dann heißt das durch obige Konstruktion gewonnen $GL(n,\K)$-
Hauptfaserbündel $(\mathfrak{B}_V, \tilde \pi, B)$ \textbf{Rahmenbündel} von $B$.

Ist $(B,g)$ eine Riemansche Mannigfaltigkeit und sind in der Konstruktion die $\mathfrak{B}_b$ zusätzlich orthonormal gewählt, so heißt das entstehende $O(n)$-Hauptfaserbündel \textbf{orthonormales
Rahmenbündel}.}

Natürlicherweise ergeben sich folgende $G$-Hauptfaserbündel als Bündel (geordneter) Basen:

\begin{table}[hbt]
\centering
  \begin{tabular}{c|c|c|c}
    $\K$       & Vektorbündel & (geordnete) Basis & $G$         \\
    \hline
    $\R, \ \C$ & beliebig     & alle              & $GL(n,\K)$  \\
    $\R$       & Riemannsch   & orthonormal       & $O(n)$      \\
    $\C$       & Hermitisch   & orthonormal       & $U(n)$      \\
    $\R$       & orientiert   & pos. orientiert   & $GL^+(n,\R)$\\
    $\R$ & Riemannsch, orientiert & orthonormal, orientiert & $SO(n)$ \\
  \end{tabular}
\end{table}

Im Folgenden werden noch einige weitere Beispiele für Hauptfaserbündel gegeben:

\thmbox{bsp}{Triviales Hauptfaserbündel}{Für alle Differenzierbaren Mannigfaltigkeiten $B$ und Lie-Gruppen $G$ ist $(B \times G, \pi_B, B; G)$ das triviale $G$-Hauptfaserbündel.}

\thmbox{bsp}{Homogenes Bündel}{Sei $G$ eine Lie-Gruppe, $H \subset G$ eine abgeschlossene Untergruppe, $\fak{G}{H}$ der zugehörige homogene Raum und $\pi \colon G \to \fak{G}{H}$ die Projektion auf den Faktorraum.
Dann ist $(G, \pi, \fak{G}{H}; H)$ ein $H$-Hauptfaserbündel, genannt das homogene Bündel.}

\thmbox{bsp}{Steifel- und Grassmann-Mannigfaltigkeit}{Für $\K \in \{ \R, \C, \mathbb{H} \}$ \footnote{$\mathbb{H}$ bezeichnet den Quaternionenkörper.} bezeichne $<\cdot,\cdot>_{\K^n}$ das 
Standard-Skalarprodukt.

Unter der Stiefelmannigfaltigkeit $V_k(\K^n)$ versteht man die Menge aller $k$-Tupel von orthonormalen Vektoren im $\K^n$:
\[ V_k(\K^n) := \{ (v1, \dots, v_k) \ | \ v_i \in \K^n, \ <v_i,v_j>_{\K^n} = \delta_{ij} \footnotemark \}\] \footnotetext{$\delta_{ij} \colon \N \times \N \to \{0,1\}$ mit $\delta_{ij} = 1 
\Leftrightarrow i = j$ bezeichnet das Kronecker-Delta.} versehen mit einer Mannigfaltigkeitsstruktur.

Unter der Grassmann-Mannigfaltigkeit $G_k(\K^n)$ versteht man die Menge aller $k$-dimensionalen Untervektorräume des $\K^n$ versehen mit einer Mannigfaltigkeitsstruktur. 

Durch
\[ \pi \colon V_k(\K^n) \to G_k(\K^n), (v_1, \dots, v_k) \mapsto span(v_1, \dots, v_k) \]
wird $\left(V_k(\K^n), \pi, G_k(\K^n)\right)$ zu einem $\mathfrak{G}$-Hauptfaserbündel. Die typische Faser $\mathfrak{G}$ ist dabei abhängig vom Körper $\K$:
\begin{table}[h]
\centering
  \begin{tabular}{c|c}
    $\K$ & $\mathfrak{G}$\\
    \hline
    $\R$ & $O(n)$\\
    $\C$ & $U(n)$\\
    $\mathbb{H}$ & $Sp(n)$\\
  \end{tabular}
\end{table}}