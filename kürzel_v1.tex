\documentclass{llncs}

\usepackage[utf8]{inputenc}

%\usepackage{fixltx2e}
%\usepackage[T1]{fontenc}
%\usepackage{textcomp}
\usepackage[german]{babel}		% english or german

% references
\usepackage[backend=biber,isbn=false,doi=false]{biblatex}
\addbibresource{template.bib}
	%\usepackage[numbers]{natbib}
	%\bibliographystyle{abbrvnat}
	%\makeatletter
	%	\renewcommand\bibsection{\section*{\refname\@mkboth{\MakeUppercase{\refname}}{\MakeUppercase{\refname}}}}
	%\makeatother

\usepackage{xcolor}
\usepackage{todonotes}

\usepackage{pgf}
\usepackage{tikz}
\usetikzlibrary{arrows,automata}
\usepackage{verbatim}
\tikzset{
	MyPersp/.style={scale=1.3,x={(-0.8cm,-0.2cm)},y={(0.8cm,-0.2cm)},z={(0cm,1cm)}},
% 	MyPersp/.style={scale=1.5,x={(-0.8cm,-0.4cm)},y={(1cm,0cm)},     z={(0cm,1cm)}}, % uncomment the two lines to get a lateral view
	MyPoints/.style={fill=white,draw=black,thick}
		}

\usepackage{amsmath}
\usepackage{amssymb}
\usepackage{mathtools}

\let\proof\relax
\let\endproof\relax
\usepackage{amsthm}
\theoremstyle{plain}
\spnewtheorem{thm}{Theorem}[section]{\bfseries}{\itshape} % reset theorem numbering for each section
\numberwithin{thm}{section}

\theoremstyle{definition}
\spnewtheorem{defn}[thm]{Definition}{\bfseries}{\itshape} % definition numbers are dependent on theorem numbers
\spnewtheorem{bsp}[thm]{Beispiel}{\bfseries}{\itshape} % same for example numbers
\spnewtheorem{bem}[thm]{Bemerkung}{\bfseries}{\itshape}
\spnewtheorem{lemdef}[thm]{Lemma und Definition}{\bfseries}{\itshape}
\spnewtheorem{defbem}[thm]{Definition und Bemerkung}{\bfseries}{\itshape}
\spnewtheorem{lem}[thm]{Lemma}{\bfseries}{\itshape}
\spnewtheorem{ann}{Annahme}{\bfseries}{\itshape}

\newcommand{\thmbox}[3]{
  \begin{#1}
    \textbf{#2}\\
    #3
  \end{#1}
}

\newcommand{\difman}{Differenzierbare Mannigfaltigkeit}
\newcommand{\difmans}{\difman en}

\newcommand{\TODO}[1]{\todo[inline]{#1}}

\newcommand{\norm}[1]{\left\lVert#1\right\rVert}
\newcommand{\fak}[2]{{}^{#1} / _{#2}}
\newcommand{\R}{\mathbb{R}}
\newcommand{\Q}{\mathbb{Q}}
\newcommand{\Z}{\mathbb{Z}}
\newcommand{\N}{\mathbb{N}}
\newcommand{\ttit}[1]{\text{\textit{#1}}}
\newcommand{\pot}[1]{\mathcal{P}(#1)}
\newcommand{\after}{\circ}

%\usepackage{nameref}
%\usepackage{varioref}

% Figures, graphics etc.
\usepackage{graphicx}
\DeclareGraphicsExtensions{.pdf,.jpg,.png}
\usepackage{subcaption}
\usepackage{listings}
\lstset{
	escapechar=\%, 
	basicstyle=\footnotesize,
	frame=tb, 
	breaklines=true,
	numbers=left,
	numbersep=6pt,
	tabsize=3,
}


\usepackage[
	pdfborder={0 0 0},
	pdfusetitle,
	colorlinks=false,
	bookmarksnumbered=true,
	bookmarksopenlevel=1,
	bookmarksopen=true,
	bookmarksdepth=3
]{hyperref}

% llncs hyperref fix
\makeatletter
\providecommand*{\toclevel@author}{0}
\providecommand*{\toclevel@title}{0}
\makeatother

\addto\extrasenglish{%
	\renewcommand{\sectionautorefname}{Section}%
}

\let\subsectionautorefname\sectionautorefname
\let\subsubsectionautorefname\sectionautorefname


%%%%%%%%%%%%%%%%%%%%%%%%%%%%%%%%%%%%%%%%%%%%%%%%%%%%%%%%%%%%%%%%%%%%%%%%%%%%%%%
%%% BEGIN DOCUMENT
%%%%%%%%%%%%%%%%%%%%%%%%%%%%%%%%%%%%%%%%%%%%%%%%%%%%%%%%%%%%%%%%%%%%%%%%%%%%%%%
\title{Hauptfaserbündel und Vektorbündel}
%\subtitle{My (optional) Subtitle}
\author{Adrian Pegler}
\institute{Christian-Albrechts-Universität zu Kiel\\Arbeitsgruppe Geometrie\\24098 Kiel}

\begin{document}

\maketitle
\begin{center}
	\today
\end{center}

\begin{abstract}
  This is my abstract.
\end{abstract}

%\section{Introduction}

\section{Faserbündel}
\label{sec:faserbundel}

\thmbox{defn}{Faserbündel}{Hier folgt die Definition}

\thmbox{bem}{}{Muhaha}


\section{Hauptfaser- und Rahmenbündel}
\label{sec:vektorbundel}

\thmbox{defn}{Hauptfaserbündel}{Sei $(P, \pi, B)$ ein Faserbündel und $G$ eine Lie-Gruppe. $(P, \pi, B) = (P, \pi, B ; G)$ heißt \textbf{$G$-Hauptfaserbündel}, falls gilt:
\begin{enumerate}
 \item $G$ wirkt von rechts als Liesche Transformationsgruppe auf $P$. Die Wirkung ist frei, fasertreu und faserweise transitiv.
 \item Es gibt einen Bündelatlas $\{ \Phi_U \}$ aus G-äquivarianten (lokalen) Trivialisierungen. Das heißt:
      \begin{enumerate}
       \item $\Phi_U \colon \pi^{-1}(U) \to U \times G$ ist ein Diffeomorphismus.
       \item $\pi_U \after \Phi_U = \pi$.\footnotemark
       \item $\Phi_U (p \cdot g) = \Phi_U(p) \cdot g$ für alle $p \in \pi^{-1}$, $g \in G$.
      \end{enumerate}\footnotetext{$\pi_u$ ist die Projektion $U \times G \to U$.}
\end{enumerate}
Es gilt also folgendes Schema:
\begin{center}
  \begin{tikzpicture}[->, >=stealth', shorten >=1pt, auto, node distance=2.8cm, semithick]
    \tikzstyle{every state}=[draw=none]

    \node[state] (A)			{$\pi^{-1}(U) \times G$};
    \node[state] (B) [below right of=A]	{$U \times G \times G$};
    \node[state] (C) [above right of=B] {$\pi^{-1}(U)$};
    \node[state] (D) [below right of=C] {$U \times G$};

    \path (A) edge      node {$\cdot_G$} 		(C)
	  (C) edge      node {$\pi_U$} 			(D)
	  (A) edge      node {$\Phi_U \times id_G$} 	(B)
	  (B) edge	node {$id_U \times \cdot_G$}	(D);
  \end{tikzpicture}
\end{center}%\\
Die Lie-Gruppe $G$ heißt auch \textbf{Strukturgruppe} des Haputfaserbündels.
}

% \section{Conclusions}
% \label{sec:conclusions}


\printbibliography

\end{document}
