\documentclass{llncs}

\usepackage[utf8]{inputenc}

%\usepackage{fixltx2e}
%\usepackage[T1]{fontenc}
%\usepackage{textcomp}
\usepackage[english]{babel}		% english or german

% references
\usepackage[backend=biber,isbn=false,doi=false]{biblatex}
\addbibresource{template.bib}
	%\usepackage[numbers]{natbib}
	%\bibliographystyle{abbrvnat}
	%\makeatletter
	%	\renewcommand\bibsection{\section*{\refname\@mkboth{\MakeUppercase{\refname}}{\MakeUppercase{\refname}}}}
	%\makeatother

\usepackage{xcolor}
\usepackage{todonotes}

%\usepackage{nameref}
%\usepackage{varioref}

% Figures, graphics etc.
\usepackage{graphicx}
\DeclareGraphicsExtensions{.pdf,.jpg,.png}
\usepackage{subcaption}
\usepackage{listings}
\lstset{
	escapechar=\%, 
	basicstyle=\footnotesize,
	frame=tb, 
	breaklines=true,
	numbers=left,
	numbersep=6pt,
	tabsize=3,
}


\usepackage[
	pdfborder={0 0 0},
	pdfusetitle,
	colorlinks=false,
	bookmarksnumbered=true,
	bookmarksopenlevel=1,
	bookmarksopen=true,
	bookmarksdepth=3
]{hyperref}

% llncs hyperref fix
\makeatletter
\providecommand*{\toclevel@author}{0}
\providecommand*{\toclevel@title}{0}
\makeatother

\addto\extrasenglish{%
	\renewcommand{\sectionautorefname}{Section}%
}

\let\subsectionautorefname\sectionautorefname
\let\subsubsectionautorefname\sectionautorefname


%%%%%%%%%%%%%%%%%%%%%%%%%%%%%%%%%%%%%%%%%%%%%%%%%%%%%%%%%%%%%%%%%%%%%%%%%%%%%%%
%%% BEGIN DOCUMENT
%%%%%%%%%%%%%%%%%%%%%%%%%%%%%%%%%%%%%%%%%%%%%%%%%%%%%%%%%%%%%%%%%%%%%%%%%%%%%%%
\title{My Title}
\subtitle{My (optional) Subtitle}
\author{The Author}
\institute{Kiel University\\Department of Computer Science\\24098 Kiel, Germany}

\begin{document}

\maketitle
\begin{center}
	\today
\end{center}

\begin{abstract}
  This is my abstract.
\end{abstract}

\section{Introduction}

  This is my introduction. \textcite{Shaw2003WritingGoodSoftwareEngineeringResearchPapersMinitutorial} wrote a paper with hints on how to write good software engineering research papers. By the way, this was an example for using the \textit{biblatex} command \texttt{\textbackslash{}textcite\{\}}.

  \autoref{sec:anotherSection} presents everything one must know. 
  The conclusions follow in Section~\ref{sec:conclusions}.

\section{Another Section}\label{sec:anotherSection}

  \textit{AspectJ} can be used to weave cross-cutting concerns into Java programs \cite{AspectJ2007}. 
  By the way, this was an example for using the \textit{biblatex} command \texttt{\textbackslash{}cite\{\}}.

  We will now demonstrate how to use subfigures (see Figure~\ref{fig:subfig}). \autoref{fig:circle} shows a circle.
  A star is displayed in \autoref{fig:star}.

  \begin{figure}
    \begin{subfigure}[l]{.49\textwidth}
	  \centering
      \includegraphics[width=0.3\textwidth]{template_circle.pdf}
      \caption{left figure.}
      \label{fig:circle}
	\end{subfigure}
	\hfill
    \begin{subfigure}[c]{.49\textwidth}
      \centering
      \includegraphics[width=0.3\textwidth]{template_star.pdf}
      \caption{right figure.}
      \label{fig:star}
	\end{subfigure}
    \caption{A circle (a) and a star (b). Note that any caption ends with a full stop character.}
    \label{fig:subfig}
  \end{figure}

\section{Conclusions}\label{sec:conclusions}

  These are my conclusions.

\printbibliography

\end{document}
