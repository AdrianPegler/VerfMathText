\section{Faserbündel}
\label{sec:faserbundel}

\thmbox{defn}{Faserbündel}{Seien $E$, $B$, $F$ \difmans, $\pi \colon E \to B$ eine glatte surjektive Funktion.\\
Falls es um jeden Punkt $x \in B$ eine Umgebung $U$ sowie einen Diffeomorphismus $\Phi_U \colon \pi^{-1}(U) \to U \times F$ gibt, sodass 
\begin{equation}
\pi_U \after \Phi_U = \pi
\label{eq:kommut}
\end{equation}
gilt, nennen wir $\left( E, \pi, B \right)$ \textbf{Faserbündel} mit \textbf{typischer Faser} $F$. Nach \autoref{eq:kommut} kommutiert also folgendes Schema:
\begin{center}
  \begin{tikzpicture}[->, >=stealth', shorten >=1pt, auto, node distance=2.8cm, semithick]
    \tikzstyle{every state}=[draw=none]

    \node[state] (A)			{$\pi^{-1}(U) \subset E$};
    \node[state] (B) [below right of=A]	{$U \subset B$};
    \node[state] (C) [above right of=B] 	{$U \times F$};

    \path (A) edge              node {$\Phi_U$} (C)
	      edge              node {$\pi$} (B)
	  (C) edge              node {$\pi_U$} (B);
  \end{tikzpicture}
\end{center}%\\
$B$ heißt \textbf{Basisraum} und $E$ \textbf{Totalraum} des Faserbündels. Die Abbildung $\Phi_U$ wird auch \textbf{lokale Trivialisierung} oder \textbf{Bündelkarte} genannt.
Mit $E_x := \pi^{-1}(x)$  bezeichnen wir für alle $x \in B$ die \textbf{Faser} über $x$. }

\begin{figure}[!b]
\centering
  \begin{tikzpicture}[MyPersp]%,font=\large]
    % the base circle is the unit circle in plane Oxy
    \def\h{1.5}% Heigth of the ellipse center (on the axis of the cylinder)
    \def\offa{0.5}% 
    \def\offb{2.6}%
    \def\offc{1.4}%
    \def\offd{3}%

    %%%%%%%%%%%%%%%%%%%%%
    % representation of E
    %%%%%%%%%%%%%%%%%%%%%
    \foreach \t in {10,20,...,360}% generatrices 1
      \draw[dashed] ({\offb*cos(\t)},{\offc*sin(\t)},0)--({\offb*cos(\t)},{\offc*sin(\t)},{2.0*\h});
    \draw[very thick] (\offb*1,\offc*0,0) % lower circle
	\foreach \t in {5,10,...,360}{--({\offb*cos(\t)},{\offc*sin(\t)},0)}--cycle;
    \draw[very thick] (\offb*1,\offc*0,{2*\h}) % upper circle
	\foreach \t in {5,10,...,360}{--({\offb*cos(\t)},{\offc*sin(\t)},{2*\h})}--cycle;
	
    \fill[] (-1.7,0.25,0) circle (0pt) node[left]{$E$}; 
	
    %%%%%%%%%%%%%%%%%%%%%
    % representation of B, U and x
    %%%%%%%%%%%%%%%%%%%%%
    \draw[very thick] (\offb*1,\offc*0,{-\h}) % B
	\foreach \t in {5,10,...,360}{--({\offb*cos(\t)},{\offc*sin(\t)},{-\h})}--cycle;
    \draw[magenta, very thick] ({1+\offa},{0},{-\h}) % U
	\foreach \t in {5,10,...,360}{--({cos(\t)+\offa},{sin(\t)},{-\h})}--cycle;
    
    \fill[]     (-1.7,0.25,-\h) circle (0pt) node[left]{$B$}; 
    \fill[magenta] (1,0,-\h) circle (0pt)node[right]{$U$};
    \fill[blue]	 (0,0,-\h) circle (1pt)node[left]{$x$};
    
  %   \draw[<-,thick] (-1.7,0.25,-\h)--(-1.7,0.25,2*\h) node[above right]{$\pi$};
    \draw[<-,thick] (-1.7,0.25,-\h)--(-1.7,0.25,0);
    \fill[] (-1.7,0.25,-0.5*\h) circle (0pt)node[right]{$\pi$};
    
    %%%%%%%%%%%%%%%%%%%%%
    % representation of pi^-1(U)
    %%%%%%%%%%%%%%%%%%%%%
    \foreach \t in {20,40,...,360}% generatrices
      \draw[magenta, dashed] ({cos(\t)+\offa},{sin(\t)},0)--({cos(\t)+\offa},{sin(\t)},{2.0*\h});
    \draw[magenta, very thick] ({1+\offa},{0},0) % lower circle
	\foreach \t in {5,10,...,360}{--({cos(\t)+\offa},{sin(\t)},0)}--cycle;
    \draw[magenta, very thick] ({1+\offa},{0},{2*\h}) % upper circle
	\foreach \t in {5,10,...,360}{--({cos(\t)+\offa},{sin(\t)},{2*\h})}--cycle;
	
    \draw[blue, very thick] (0,0,0) -- (0,0,2*\h) node[left]{$E_x$};
  %   \fill[blue] (0,0,\h) circle (0pt) node[left]{$\pi^{-1}(x)$};
    
    %%%%%%%%%%%%%%%%%%%%%
    % representation of UxF
    %%%%%%%%%%%%%%%%%%%%%
    \foreach \t in {20,40,...,360}% generatrices
      \draw[orange, dashed] ({cos(\t)-\offd},{sin(\t)+\offd},0)--({cos(\t)-\offd},{sin(\t)+\offd},{2.0*\h});
    \draw[orange, very thick] ({1-\offd},{0+\offd},0) % lower circle
	\foreach \t in {5,10,...,360}{--({cos(\t)-\offd},{sin(\t)+\offd},0)}--cycle;
    \draw[orange, very thick] ({1-\offd},{0+\offd},{2*\h}) % upper circle
	\foreach \t in {5,10,...,360}{--({cos(\t)-\offd},{sin(\t)+\offd},{2*\h})}--cycle;
	
    \draw[red, very thick] (0-\offd,0+\offd,0) -- (0-\offd,0+\offd,2*\h);
    \fill[red]    (0-\offd,0+\offd,\h) circle (0pt) node[right]{$F$};
    \fill[orange] (0-\offd,0+\offd,2*\h) circle (0pt) node[left]{$U \times F$}; 
    
    %%%%%%%%%%%%%%%%%%%%%
    % representation of Phi_U
    %%%%%%%%%%%%%%%%%%%%%
    \draw[->,thick] (-0.5,0.5,\h) -- (-2.2,2.2,\h) node[above left]{$\Phi_U$};

  \end{tikzpicture}
  \caption{Skizze eines Faserbündels.}
  \label{fig:faserbundel}
\end{figure}

Diese Definition ist in \autoref{fig:faserbundel} veranschaulicht. Neben der Beziehung zwischen dem Totalraum $E$ und dem Basisraum $B$, sind in blau ein Punkt $x \in B$ und sein Urbild $E_x$
bezüglich $\pi$ sowie in magenta die Umgebung $U$ um $x$ und deren Urbild unter $\pi$ dargestellt. Zudem sind die typische Faser $F$ in rot und der Diffeomorphismus $\Phi_U$ sowie dessen Bild 
$U \times F$ in orange abgebildet.

\thmbox{defbem}{}{Für alle $x \in B$ sei folgende Abbildung definiert: \[\Phi_{U,x} \colon E_x \to F , \ \Phi_{U,x} := \pi_F \after \Phi_U.\] 
Dann ist für jede Faser $E_x$ durch $\Phi_{U,x}$ ein Isomorphismus auf die typische Faser $F$ gegeben, was der Grund für deren Benennung ist.}

% \thmbox{defbem}{Übergangsfunktionen und Cozykelbedingung}{Seien $U_i, U_k \subset B$ offene Umgebungen mit Bündelkarten $\Phi_{U_i}$ und $\Phi_{U_k}$. 
% }

\thmbox{defn}{Isomorphie}{Seien $( E, \pi, B)$ und $( \tilde E, \tilde \pi, B )$ Faserbündel über dem gleichen Basisraum $B$. Wir nennen die Faserbündel isomorph, falls
es einen Diffeomorphismus $\Psi \colon E \to \tilde E$ gibt, sodass $\tilde \pi \after \Psi = \pi$. Schematisch gilt also Folgendes:
\begin{center}
  \begin{tikzpicture}[->, >=stealth', shorten >=1pt, auto, node distance=2.8cm, semithick]
    \tikzstyle{every state}=[draw=none]

    \node[state] (A)			{$E$};
    \node[state] (B) [below right of=A]	{$B$};
    \node[state] (C) [above right of=B] {$\tilde E$};

    \path (A) edge              node {$\Psi$} (C)
	      edge              node {$\pi$} (B)
	  (C) edge              node {$\tilde \pi$} (B);
  \end{tikzpicture}
\end{center}}%

\thmbox{bsp}{}{Seien $M$ und $F$ \difmans, $\pi \colon M \times F \to M$ die Projektion auf die erste Komponente. Dann ist $(M \times F, \pi, M)$ ein Faserbündel.
Jedes zu diesem Faserbündel isomorphe Faserbündel nennen wir \textbf{trivial}.}

\thmbox{bsp}{}{Sei $M$ eine n-dimensionale \difman. Die Folgenden Bündel sind Faserbündel mit Basisraum $M$:
\begin{enumerate}
 \item das Tangentialbündel: $(TM, \pi, M); \R^n$,
 \item das Cotangentialbündel: $(T^* M, \pi, M); \R^n$,
 \item das k-Formenbündel: $(\Omega^k M, \pi, M); \R^{\binom{n}{k}}$.
\end{enumerate}
}

\thmbox{bsp}{}{Sei $(B,g)$ eine n-dimensionale Riemansche Mannigfaltigkeit. Setze: 
\begin{align*}
  E   &:= \{ x \in TB \ | \ \norm{x}_g = 1 \},\\
  \pi &:= {\pi_{_{TB}}\big|}_E.
\end{align*}
Das Faserbündel $(E,\pi,B)$ heißt \textbf{Einheits-Sphären-Bündel}.

Dieser Name leitet sich von der typischen Faser $F = S^{n-1}$ her. Dies folgt, da mit den Definitionen von $E$ und $\pi$ für alle $p \in B$ gilt: 
\[ \pi^{-1}(p) = \{ x \in T_pB \ | \ \norm{x}_g = 1 \}.\]
Die lokalen Trivialisierungen erhält man aus denen des Tangentialbündels. 
}

\thmbox{lemdef}{Rückzug}{ Seien $M, N$ \difmans, und $f \colon M \to N$ eine glatte Abbildung. Für ein Faserbündel $(E, \pi, M)$ definieren wir $f^*(E, \pi, M) := (f^*E, \tilde \pi, N)$ durch:
\begin{align*}
  f^*E &:= \{ (y,e) \in N \times E \ | \ f(y) = \pi(e) \}\\
  \tilde \pi(y,e) &:= y
\end{align*}
Damit ist $(f^*E, \tilde \pi, N)$ ein Faserbündel über $N$ und heißt der \textbf{Rückzug} (engl.: pull back) von $(E, \pi, M)$ entlang $f$. Dabei bleibt die typische Faser erhalten.}

\thmbox{bem}{}{Sei $F$ eine glatte Mannigfaltigkeit und $\varphi \colon F \to F$ ein Diffeomorphismus. Dann wirkt $\Z$ durch $(k, (t,f)) \mapsto (t+k, \varphi^k(f))$ eigentlich diskontinuierlich auf 
$\R \times F$.

Mit $E := \fak{\R \times F}{\Z}$ und $\pi \colon E \to \fak{\R}{\Z} \cong S^1 =: B$ ist $(E, \pi, B)$ ein Faserbündel mit typischer Faser $F$.}

Geometrisch wird der Totalraum $E$ also aus den trivialen Bündeln $[0,1] \times F \to [0,1]$ durch ``zusammenkleben'' der Fasern durch den Diffeomorphismus $\varphi$ konstruiert.
Um lokale Trivialisierungen zu konstruieren, nutzt man die (globale) Trivialität der Projektion $\pi_\R \colon \R \times F \to \R$ sowie die eigentliche Diskontinuität der Wirkung.

\thmbox{defn}{Schnitt}{Unter einem glatten Schnitt eines Faserbündels $(E, \pi, B)$ versteht man eine glatte Abbildung $s \colon B \to E$, sodass $\pi \after s = id_B$. Mit $\Gamma(E)$ bezeichnen
wir die Menge aller Schnitte in $E$.}

\thmbox{bem}{}{Für ein triviales Faserbündel $(M \times F, \pi_M, M)$ gilt offensichtlich:
\[\Gamma(M \times F) = \mathcal{C}^\infty(M,F)\]
Weitere bekannte Objekte aus der Geometrie, die als Schnitte von Faserbündeln aufgefasst werden können sind:
}